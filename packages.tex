%%-------------------Emacs PostScript "pretty-print" page width (97 columns)-------------------%%
%% Math packages-----------------%-------------------------------
\usepackage{amsmath}
%\usepackage{amssymb,mathrsfs}%provides symbols and script math (conflicts with mathdesign fonts)

%% Text packages-----------------%-------------------------------
\usepackage{textcomp,lmodern}%for special symbols 
\usepackage[utf8]{inputenc}
\usepackage[T1]{fontenc}%modern font encoding

%% The default font is Latin Modern (replaces Computer Modern).  Following are a number of fonts 
%  with extensive math and symbol support.  To use a particular font, uncomment the associated
%  line.  If mathdesign fonts are used, the amssymb and mathrsfs packages should not be loaded
%  independently.

%% PostScript Fonts--------------%-------------------------------
%\usepackage[osf]{mathpazo}  %URW Palladio (A Palatino/Book Antiqua clone), an old style serif
%\usepackage{mathptmx}       %URW Nimbus Roman (A Times clone), a transitional serif typefaces
%\usepackage[scaled]{helvet} %URW Nimbus Sans (A Helvetica clone), a sans-serif typeface 
%\renewcommand*\familydefault{\sfdefault} % set base font (body text) to sans serif

%% Math Design Fonts-------------%-------------------------------
%\usepackage[urw-garamond]{mathdesign}			%Old style serif typeface
% \usepackage[adobe-utopia]{mathdesign}			%Transitional serif typefaces (professional)
%\usepackage[bitstream-charter]{mathdesign}	%Glyphic serif typeface optimized for printing

%% The following lines of code assume that the graphics files are .eps files located in a
%  directory named Figures.  Black and white versions of color graphics files end in _bw.  For
%  example, the two files
%			/Figures/foo.eps
%			/Figures/foo_bw.eps
%  may be included with a single command, e.g., \includegraphics{fig2}.  The black and white
%  figures are then selected by uncommenting the declare graphics extension rule below.

%% Figure packages---------------%-------------------------------
\usepackage{ifpdf}
\ifpdf
	\usepackage[pdftex]{graphicx}
	\usepackage[update]{epstopdf}
	%May need to include -enable-write18 in command line arguments to make PDF figures (deprecated)
	\epstopdfDeclareGraphicsRule{.eps}{pdf}{.pdf}{ps2pdf -dEPSCrop #1 \OutputFile}
	\epstopdfDeclareGraphicsRule{_bw.eps}{pdf}{_bw.pdf}{ps2pdf -dEPSCrop #1 \OutputFile}
	\epstopdfsetup{suffix=}%
\else
	\usepackage{graphicx}	
	\usepackage[all,light]{draftcopy}% Places "DRAFT" on each page.  Only works with PS output.	
\fi
%\DeclareGraphicsExtensions{_bw.eps,.eps}	%Turn on/off BW figures, assumes BW figures end in _bw.eps
\graphicspath{{Figures/}}%set graphics path(s) here

%% Table packages----------------%-------------------------------
\usepackage{dcolumn}% Align table columns on decimal point
\newcolumntype{d}[1]{D{.}{.}{#1}}
\newcolumntype{.}{D{.}{.}{-1}}
\newcolumntype{,}{D{,}{,}{-1}}
\newcolumntype{)}{D{)}{)}{-1}}

\renewcommand{\arraystretch}{1.3} % enlarge line spacing

\usepackage{multirow}

%% Formatting packages-----------%-------------------------------
\usepackage{hyperref}
\hypersetup{%
	pdfstartpage=1,                %Opening page number (absolute)
	pdfstartview=FitV,             %Fits the horiz. width in the window (FitV for vertical)
	%pdfstartview={XYZ null null 1},%view page at 100%
	bookmarksopen=true,            %Displays Bookmarks in the Navigation Panel
	bookmarksopenlevel=0,          %\maxdimen all levels, 0 chapters, 1 sections
	bookmarksnumbered=true,        %Numbers bookmarks with section numbers
	final=true,                    %keeps hyperref features in draft mode
	colorlinks=true,               %colors the links instead of using boxes
	urlcolor=blue,                 %makes URL hyperlinks blue (instead of pink)
	linkcolor=black,               %makes internal links black (instead of red)
	citecolor=black,               %makes citation links black (instead of green)
}
\usepackage{fancyvrb}%allows verbatim in footnotes (and makes hyperref work with footnotes)
\usepackage{indentfirst}%indents first paragraph of each section

%% Formatting packages-----------%-------------------------------
\usepackage[numbers,sort&compress]{natbib}
